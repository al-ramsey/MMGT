\documentclass{MetricNotes2023}

% Language setting
% Replace `english' with e.g. `spanish' to change the document language


% Set page size and margins
% Replace `letterpaper' with `a4paper' for UK/EU standard size
\usepackage[a4paper,top=2cm,bottom=2cm,left=2cm,right=2cm,marginparwidth=1.75cm]{geometry}

% Useful packages
\usepackage{graphicx}
\graphicspath{ {./images/} }
\usepackage{float}


%inserting a figure:

%\begin{figure}[h]
%\centering
%\includegraphics[width=0.5\textwidth]{pfigure_1}
%\caption{The compound pendulum}
%\end{figure}
\setcounter{MaxMatrixCols}{20}
\usepackage{amsmath}
\usepackage{enumitem}
\usepackage{amssymb}
\usepackage{amsthm}
\usepackage{realhats} %for hats
\usepackage[colorlinks=true, allcolors=blue]{hyperref}
%\usepackage{hyperref}
\usepackage{mathtools}
\usepackage{parskip}
%\usepackage{mbboard}
%\usepackage[mtpbbi]{mtpro2}
\usepackage{mathrsfs}


% knots stuff

\usepackage{tikz}
\usetikzlibrary{cd}
\usetikzlibrary{shapes,snakes}
\usepackage{listings}

%\graphicspath{ {figs/} }

\usepackage{cleveref}
\newcommand{\notimplies}{%
  \mathrel{{\ooalign{\hidewidth$\not\phantom{=}$\hidewidth\cr$\implies$}}}}

%keyword: edit

\def\summing{\ensuremath\sum^{\infty}_{n=1}}
\def\summingo{\ensuremath\sum^{\infty}_{n=0}}
\def\bb{\ensuremath\mathbb}
\def\la{\ensuremath\langle}
\def\ra{\ensuremath\rangle}
\def\notto{\ensuremath\nrightarrow}
\def\subq{\ensuremath\subseteq}

\def\met{\ensuremath(X, d)}
\def\metx{\ensuremath(X, d_X)}
\def\mety{\ensuremath(Y, d_Y)}
\def\topox{\ensuremath(X, \mathcal{T}_X)}
\def\topoy{\ensuremath(Y, \mathcal{T}_Y)}

\def\comp{\ensuremath\mathbb{C}}
\def\real{\ensuremath\mathbb{R}}
\def\rat{\ensuremath\mathbb{Q}}
\def\inte{\ensuremath\mathbb{Z}}
\def\nat{\ensuremath\mathbb{N}}
\def\topo{\ensuremath\mathcal{T}}

\def\sequiv{\ensuremath\stackrel{\text{S}}{\sim}}
\def\rr{\ensuremath\mathcal{R}}
\def\com{\ensuremath\Delta}
\def\id{\ensuremath\text{id}}
\def\hopf{\ensuremath\mathcal{H}}
\def\psip{\ensuremath\psi_{+}}
\def\psin{\ensuremath\psi_{-}}
\def\psipm{\ensuremath\psi_{\pm}}
\DeclareMathOperator{\trace}{trace}
\DeclareMathOperator{\tr}{tr}
\DeclareMathOperator{\str}{str}
\DeclareMathOperator{\End}{End}
\DeclareMathOperator{\Hom}{Hom}
\DeclareMathOperator{\lk}{lk}
\DeclareMathOperator{\ob}{ob}
\DeclareMathOperator{\Fin}{Fin}
\DeclareMathOperator{\Ch}{Ch}

\def\done{\begin{flushright}\vspace{-4.35ex}\(\qed\)\end{flushright}}

\def\mlip{\ensuremath\abs f_\text{Lip}}


%\DeclareSymbolFont{bbold}{U}{bbold}{m}{n}
%\DeclareSymbolFontAlphabet{\mathbbold}{bbold}

% \limits\sum or \sum\limits (one of them) will put the text under the sum
% [upquote=true] on \begin{lstlisting} will stop backticks being interpreted as quotes

\usepackage[backend=bibtex]{biblatex}
\usepackage{csquotes}
\addbibresource{References.bib}

\author{Ali Ramsey}
\title{Symmetric Monoidal \(\infty\)-Categories}
\date{\vspace{-5ex}}

\counterwithin{figure}{section}
\newcommand{\circo}{~\raisebox{1pt}{\tikz \draw[line width=0.6pt] circle(1.1pt);}~}

\begin{document}
\maketitle
%\input{titlepage.tex}

\DeclarePairedDelimiter{\norm}{\lVert}{\rVert} 
\DeclarePairedDelimiter{\abs}{\lvert}{\rvert} 
\DeclarePairedDelimiter{\ang}{\langle}{\rangle} 

\section{Introduction}

\begin{itemize}
\item Note: we will call \((\infty, 1)\)-categories \(\infty\)-categories. 
\item We start with an example of a pseudofunctor to motivate Grothendieck opfibrations. We explain the relationship between pseudofunctors into \textbf{Cat} and Grothendieck opfibrations and how to pass between the two.
\item We define a symmetric monoidal category \((	\mathcal{C},\otimes)\) in the usual way, then note that it can be written as a pseudofunctor to \textbf{Cat}, and thus as a Grothendieck opfibration. We construct \(\mathcal{C}^\otimes\) using the process outlined earlier, and compare it to Lurie's construction. 
\item We define symmetric monoidal functors in the usual way, and then construct the correct definitions in terms of morphisms of opfibrations.
\item We introduce inner fibrations and prove that they are stable under pullbacks and that the fibres are \(\infty\)-categories. We introduce (co)cartesian fibrations and prove that the nerve of a functor is a (co)cartesian fibration if and only if the original functor was an (op)fibration. 
\item We finally define symmetric monoidal \(\infty\)-categories, and functors between them. We give the trivial examples: the nerve of a symmetric monoidal category, and the symmetric monoidal (co)cartesian structure on an \(\infty\)-category with finite products.
\item We give an interesting example: the derived category.
\item References:
\autocite{goodfibrations}, \autocite{categorical},  \autocite{symmetricmonoidal}, \autocite{groth2015short}, \autocite{2dimensional},  \autocite{fibrational_notions},  \autocite{higheralgebra}, \autocite{lurie2008higher}, \autocite{kerodon}, \autocite{monoidalgrothendieck}. Referencing Kerodon test: \cite[\href{https://kerodon.net/tag/01UB}{Example 01UB}]{kerodon}. 
\end{itemize}

\section{Symmetric monoidal \(1\)-categories}

\subsection{The Grothendieck construction}

(Something about motivation and the pain of higher coherences. Below we give an example to complement \autocite{symmetricmonoidal}.)

Let \(\mathcal{C}\) be a category with pullbacks. Recall that for a map \(f : C \to D\) in \(\mathcal{C}\), we may define a pullback functor 
\begin{align*}
f^* : \qquad\mathcal{C}_{/D}\qquad&\to\qquad \mathcal{C}_{/C},\\
(h : X \to D) &\mapsto (f^*h : P \to C),
\end{align*}
 where we have formed a pullback 
\[\begin{tikzcd}
P \arrow[r, "h^*f"] \arrow[d, swap, "f^*h"] \arrow[dr, phantom, "\scalebox{1.3}{$\lrcorner$}" {xshift=-16pt, yshift=10pt}] & X \arrow[d, "h"]  \\
C \arrow[r, swap, "f"]  & D
\end{tikzcd}\]
in \(\mathcal{C}\). For any map 
\[\begin{tikzcd}
X \arrow[rr, "\phi"] \arrow[dr, swap, "h"]  & & X' \arrow[dl, "h'"]  \\
& D  & 
\end{tikzcd}\]
from \(h\) to \(h'\) in \(\mathcal{C}_{/D}\), we define \(f^*\phi\) to be the unique map making the diagram below commute.
\[\begin{tikzcd}
X \arrow[rr, "\phi"] \arrow[dr, swap, "h"]  & & X' \arrow[dl, "h'"]  \\
& D  & \\
& & & \\
P \arrow[rr, dashrightarrow, "f^*\phi" {xshift=-10pt}] \arrow[uuu, "h^*f"] \arrow[dr, swap, "f^*h"] & & P' \arrow[uuu, swap, "h'^*f"] \arrow[dl, "f^*h'"]\\
& C \arrow[uuu, swap, "f" {yshift=8pt}] & 
\end{tikzcd}\]
Now, we may wish to define a functor 
\begin{align*}
F : \mathcal{C}^{\text{op}} &\to \textbf{Cat},\\
C &\mapsto \mathcal{C}_{/C},
\end{align*}
which sends a map \(f : C \to D\) in \(\mathcal{C}\) to a pullback functor \(f^* : \mathcal{C}_{/D}\to \mathcal{C}_{/C}\). However a problem arises when we check that \(F\) respects composition: suppose \(f : C \to D\), \(g : D \to E\) are maps in \(\mathcal{C}\). Then 
\[F(g \circ f)(h : X \to E)=(g\circ f)^*h : P \to C,\] corresponding to the pullback
\[\begin{tikzcd}
P \arrow[r, "h^*(g\circ f)"] \arrow[d, swap, "(g \circ f)^*h"] \arrow[dr, phantom, "\scalebox{1.3}{$\lrcorner$}" {xshift=-16pt, yshift=10pt}] & X \arrow[d, "h"]  \\
C \arrow[r, swap, "g\circ f"]  & E
\end{tikzcd}\]
in \(\mathcal{C}\). On the other hand, 
\[(F(g)\circ F(f))(h : X \to E)=f^*(g^*h) : P'' \to C,\] which corresponds to the diagram below.
\[\begin{tikzcd}
P'' \arrow[r, "(g^*h)^*f"] \arrow[d, swap, "f^*(g^*h)"] \arrow[dr, phantom, "\scalebox{1.3}{$\lrcorner$}" {xshift=-16pt, yshift=10pt}] & P' \arrow[r, "h^*g"] \arrow[d, swap, "g^*h"] \arrow[dr, phantom, "\scalebox{1.3}{$\lrcorner$}" {xshift=-16pt, yshift=10pt}] & X \arrow[d, "h"]  \\
C \arrow[r, swap, "f"]  & D \arrow[r, swap, "g"] & E
\end{tikzcd}\]
The outer square is indeed a pullback square, since the inner two squares are, so we have a unique isomorphism \(P \cong P''\). However, we do not in general have equality. This is because pullbacks are only unique up to unique isomorphism, and in defining a pullback functor we made arbitrary (and not necessarily compatible) choices of \(P, P'\) and \(P''\). Thus, we have not defined a functor \(\mathcal{C}^{\text{op}}\to \textbf{Cat}\), rather, we have defined what is known as a \textit{pseudofunctor}; that is, a weak functor between 2-categories. 

The above example is just one way in which pseudofunctors into \textbf{Cat} naturally arise; another common example is the pseudofunctor
\begin{align*}
\textbf{CRing}&\to \textbf{Cat}\\
R\; &\mapsto R \textbf{-Mod},
\end{align*}
which sends a ring homomorphism \(\phi : R \to S\) to the functor \(-\otimes_R S : R \textbf{-Mod}\to S \textbf{-Mod}\); this is known as \textit{extension of scalars}. However, to give the data of a pseudofunctor \(F : \mathcal{C} \to \textbf{Cat}\), we must specify not only the functions \(\ob(\mathcal{C})\to \ob(\textbf{Cat})\) and \(\Hom_\mathcal{C}(X, Y)\to \Hom_\textbf{Cat}(F(X), F(Y))\) for each \(X, Y \in \mathcal{C}\), but also natural isomorphisms
\[F(\id_X) \cong \id_{F(X)}, \quad F(g \circ f) \cong F(g) \circ F(f).\]

Thus, as we move into the realm of higher categories, the data needed to define the appropriate notion of a functor grows rapidly. Since we intend to generalise these concepts to \(\infty\)-categories, we will turn to a different method of encoding the same information - Grothendieck fibrations and opfibrations.

\begin{definition}
Let \(p : X \to \mathcal{C}\) be a functor, let \(f : c \to c'\) be a morphism in \(\mathcal{C}\), and let \(\phi : x \to x'\) be a morphism in \(X\) lying over \(f\).

We say that \(\phi\) is \textit{cartesian} if for any other morphism \(\psi : x'' \to x'\) in \(X\), and for any morphism \(g : p(x'')\to c\) in \(\mathcal{C}\) satisfying \(f \circ g = p(\psi)\), there exists a unique morphism \(\gamma : x'' \to x\) such that \(p(\gamma)=g\) and \(\psi = \gamma \circ \phi\).

Dually, \(\phi\) is \textit{cocartesian} if for any other morphism \(\psi : x \to x''\) in \(X\), and for any morphism \(g : c' \to p(x'')\) in \(\mathcal{C}\) satisfying \(g \circ f = p(\psi)\), there exists a unique morphism \(\gamma : x' \to x''\) such that \(p(\gamma)=g\) and \(\psi = \gamma \circ \phi\).  

The left diagram below corresponds to a cartesian morphism, and the right diagram corresponds to a cocartesian morphism.

\[\begin{tikzcd}
 & & x'' \arrow[dddd, mapsto, ""] \arrow[dll, dashrightarrow, swap, "\exists ! \gamma"] \arrow[dl, "\forall \psi"] \\
x \arrow[r, swap, "\phi"] \arrow[dd, mapsto, swap, ""] & x' \arrow[dd, mapsto, swap, ""]  & \\
& & \\
c \arrow[r, "f"] & c'  & \\
 & & p(x'') \arrow[llu, "\forall g"] \arrow[lu, swap, "p(\psi)"] 
\end{tikzcd}\qquad\quad \begin{tikzcd}
 & & x'' \arrow[dddd, mapsto, ""] & \\
x \arrow[r, swap, "\phi"] \arrow[dd, mapsto, swap, ""] \arrow[urr, "\forall \psi"] & x' \arrow[dd, mapsto, swap, ""] \arrow[ur, dashrightarrow, swap, "\exists! \gamma"] & & X \arrow[dd, swap, "p"] \\
& & & \\
c \arrow[r, "f"] \arrow[rrd, swap, "p(\psi)"] & c' \arrow[rd, "\forall g"]  & & \mathcal{C}\\
 & & p(x'') & 
\end{tikzcd}\]

\end{definition}

\begin{definition}\label{opfibration}
Let \(p : X \to \mathcal{C}\) be a functor. Then \(p\) is a \textit{Grothendieck fibration} if for any morphism of \(\mathcal{C}\) and any lift of its target, there is a cartesian morphism with that target lying over it. Dually, \(p\) is a \textit{Grothendieck opfibration} if for any morphism of \(\mathcal{C}\) and any lift of its source, there  is a cocartesian morphism with that source lying over it.
\end{definition}

We will usually refer to Grothendieck (op)fibrations as just (op)fibrations for brevity. Note that in \autocite{symmetricmonoidal} and \autocite{lurie2008higher} these are referred to as \textit{(co)cartesian fibrations}; we reserve this term for the \(\infty\)-category analogue. 

\begin{remark}\label{rem:op}
A functor \(p : X \to \mathcal{C}\) is an opfibration if and only if \(p^{\text{op}} : X^{\text{op}}\to \mathcal{C}^{\text{op}}\) is a fibration. We give the definition of an opfibration explicitly, since we will be working with these more often than fibrations.  
\end{remark}

%(Slightly confusing thing: Theorem 8.3.1 of \autocite{categorical} says fibrations (not opfibrations) into \(\mathcal{C}\) are ``the same" as pseudofunctors \(\mathcal{C} \to \textbf{Cat}\). nLab says the same thing but for pseudofunctors \(\mathcal{C}^{\text{op}}\to \textbf{Cat}\). Definition 2.2.7 of \autocite{fibrational_notions} gives the definition of an opfibration \(p : E \to B\) as a fibration \(E^{\text{op}}\to B^{\text{op}}\), which I believe; they also say in Theorem 2.2.3 that pseudofunctors into \(\mathcal{C}\) are the same as pseudofunctors \(\mathcal{C}^{\text{op}}\to \textbf{Cat}\), which means that opfibrations into \(\mathcal{C}\) (which are fibrations into \(\mathcal{C}^{\text{op}}\)) are pseudofunctors \(\mathcal{C} \to \textbf{Cat}\). Um?)

\begin{theorem}[{\autocite{monoidalgrothendieck}, Thm 2.4}]\label{thm:grothendieck}
There is an equivalence of \(2\)-categories 
\[\textbf{Psd}[\mathcal{C}^{\text{op}}, \textbf{Cat}]\simeq \textbf{Fib}(\mathcal{C}),\]
where \(\textbf{Psd}[\mathcal{C}^{\text{op}}, \textbf{Cat}]\) is the \(2\)-category of pseudofunctors \(\mathcal{C}^{\text{op}}\to \textbf{Cat}\), and \(\textbf{Fib}(\mathcal{C})\hookrightarrow \textbf{Cat}_{/\mathcal{C}}\) is the \(2\)-category of fibrations into \(\mathcal{C}\). 
\end{theorem}

\begin{remark}\label{rem:ophell}
Combining \ref{rem:op} with \ref{thm:grothendieck} gives us a chain of equivalences
\[\textbf{Psd}[\mathcal{C}, \textbf{Cat}]\xrightarrow{\text{op} \circo} \textbf{Psd}[\mathcal{C}, \textbf{Cat}]\simeq \textbf{Psd}[(\mathcal{C}^{\text{op}})^{\text{op}}, \textbf{Cat}] \simeq \textbf{Fib}(\mathcal{C}^{\text{op}}) \simeq \textbf{opFib}(\mathcal{C}).\]
\end{remark}

We will not prove the theorem above (see \autocite{monoidalgrothendieck} or \autocite{2dimensional} for more details) but we will describe how to pass between \(\textbf{Psd}[\mathcal{C}, \textbf{Cat}]\) and \(\textbf{opFib}(\mathcal{C})\) by dualising\footnote{We use the equivalences in \ref{rem:ophell}; in particular we postcompose with the equivalence \(\text{op} : \textbf{Cat}\to \textbf{Cat}\). This is to ensure that we can define the pseudofunctor \(F : \mathcal{C} \to \textbf{Cat}\) to send \(c\in \mathcal{C}\) to \(p^{-1}\{c\}\) rather than \((p^{\text{op}})^{-1}\{c\}\simeq (p^{-1}\{c\})^{\text{op}}\).} the constructions in these sources.

Let \(F : \mathcal{C} \to \textbf{Cat}\) be a pseudofunctor. Define the category \(X\) as follows: the objects of \(X\) are pairs \((c, x)\), with \(c \in \mathcal{C}\), \(x \in F(c)\). A map \((c, x)\to (d, y)\) is a pair \((f, u)\), where \(f : c \to d\) is a morphism in \(\mathcal{C}\), and \(u : (Ff)(x)\to y\) is a morphism in \(F(d)\). For an object \((c, x) \in X\), the identity morphism is given by
\[(\id_c, F^0_x : F(\id_c)(x)\to x),\]
where \(F^0\) is the natural isomorphism \( F(\id_c)\cong \id_{F(c)} \). Further, given two maps \((f, u) : (c, x) \to (d, y)\) and \((g, v) : (d, y)\to (e, z)\), their composition \((g,v)\circ (f, u)\) is given by \(g \circ f\), together with the map
\[(F(g\circ f))(x) \xrightarrow{\left(F^2_{g,f}\right)_x} (Fg \circ Ff)(x) \xrightarrow{(Fg)(u)} (Fg)(y)\xrightarrow{v} z,\]

where \(F^2_{g, f}\) is the natural isomorphism \( F(g \circ f)\cong Fg \circ Ff\). 

\begin{lemma}
The forgetful functor \(p : X \to \mathcal{C}\) is an opfibration over \(\mathcal{C}\).
\end{lemma}

\begin{ourproof}
Let \(f : c \to d\) be a morphism in \(\mathcal{C}\), and let \(x\) be a lift of its source. Consider the map \((f, \id_{(Ff)(x)}) : (c, x)\to (d, (Ff)(x))\). Let \((g, \psi) : (c, x)\to (e, z)\) be a map in \(X\), and let \(h : d \to e\) be a map in \(\mathcal{C}\) such that \(g = h \circ f\). We wish to find some map \(\gamma : (Fh \circ Ff)(x) \to z\) such that the composition
\[F(h \circ f)(x)\xrightarrow{F^2_{h,f}}(Fh\circ Ff)(x)\xrightarrow{(Fh)(\id_{(Ff)(x)})}(Fh\circ Ff)(x)\xrightarrow\gamma z\]
is equal to the map
\[(Fg)(x)\xrightarrow{\psi} z.\]
Since \(h \circ f=g\), and \((Fh)(\id_{(Ff)(x)})=\id_{(Fh\circ Ff)(x)}\), we see that \(\psi = \gamma \circ F^2_{h,f}\). Since \(F^2\) is a natural isomorphism, we have \(\gamma = \psi \circ (F^2_{h,f})^{-1}\), so \(\gamma\) exists and is unique. Therefore, for every morphism in \(\mathcal{C}\) and every lift of its source, there is a cocartesian morphism lying over it; that is, \(p : X \to \mathcal{C}\) is a an opfibration. \done
\end{ourproof} 

Now, let \(p : X \to \mathcal{C}\) be an opfibration over \(\mathcal{C}\). Define a pseudofunctor 
\begin{align*}
\mathcal{C}&\to \textbf{Cat}\\
c &\mapsto p^{-1}\{c\}.
\end{align*}

For a map \(f : c \to d\) in \(\mathcal{C}\), we define a functor 
\begin{align*}
f_* : p^{-1}\{c\}&\to p^{-1}\{d\},
\end{align*}
which sends \(x \in p^{-1}\{c\}\) to the target \(x'\) of a cocartesian edge \(\phi_x : x \to x'\) lying over \(f\). Now, let \(f^*(y)=y'\), with \(\phi_y : y \to y'\), and let \(g : x \to y\) be a morphism in \(p^{-1}\{c\}\) (that is, a morphism in \(X\) lying over \(\id_c\)). Then, since \(\phi_y : y \to y'\) is a cocartesian edge, there is a unique lift \(\overline g : x' \to y'\) of \(\id_d\) making the square on the left commute, as shown in the diagram on the right. 

\[\begin{tikzcd}
x \arrow[r, "\phi_x"] \arrow[d, swap, dashrightarrow, " g"]  & x' \arrow[d, "\exists!\overline g"]  \\
y \arrow[r, swap, "\phi_y"]  & y'
\end{tikzcd}\quad\quad\quad \begin{tikzcd}
 & & y' \arrow[dddd, mapsto, ""] \\
x \arrow[urr, "\phi_y \circ g"] \arrow[r, swap, "\phi_x"] \arrow[dd, mapsto, swap, ""] & x' \arrow[ur, dashrightarrow, swap,  "\exists! \overline g"] \arrow[dd, mapsto, swap, ""]  & \\
& & \\
c \arrow[r, "f"] \arrow[rrd, swap, "f"] & d \arrow[rd, "\id_d"]  & \\
 & & d
\end{tikzcd}\]

We thus define \(f_*g=\overline{g}\) in \(p^{-1}\{d\}\). It can be shown that these data assemble into a pseudofunctor \(\mathcal{C}\to \textbf{Cat}\).

\subsection{Symmetric monoidal categories and functors}

We now move onto the main object of our consideration, symmetric monoidal categories. 

\begin{definition}
A \textit{symmetric monoidal category} is a category \(\mathcal{C}\) equipped with a bifunctor 
\[\otimes : \mathcal{C} \times \mathcal{C}\to \mathcal{C},\]
an object \(\textbf{1}\in \mathcal{C}\), and natural isomorphisms
\[\alpha : \otimes \circ (\otimes \times \id) \xrightarrow{\sim}\otimes \circ (\id \times \otimes),\quad l : \otimes \circ (\textbf{1}\times \id)\xrightarrow\sim \id, \quad r : \otimes \circ (\id \times \textbf{1})\xrightarrow\sim \id,\]
\[\tau : \otimes \xrightarrow\sim \sigma \circ \otimes,\]
where \(\sigma : \mathcal{C} \times \mathcal{C}\) swaps the order of the factors. These isomorphisms are subject to the condition that \(\tau^2=\id\), and that the diagrams below commute.
\[\begin{tikzcd}
& ((U\otimes V)\otimes W)\otimes X \arrow[ld, swap, "\alpha_{U\otimes V, W, X}"] \arrow[rd, "\alpha_{U, V, W}\otimes \id_X"]  & \\
(U\otimes V)\otimes (W\otimes X) \arrow[d, swap, "\alpha_{U, V, W\otimes X}"]  & & (U\otimes(V\otimes W))\otimes X \arrow[d, "\alpha_{U, V\otimes W, X}"]\\
U\otimes(V\otimes(W\otimes X)) & & U\otimes((V\otimes W)\otimes X) \arrow[ll, "\id\otimes \alpha_{V, W,X}"]
\end{tikzcd}\]

\[\begin{tikzcd}
(V\otimes \textbf{1})\otimes W \arrow[rr, "\alpha_{V, \textbf{1}, W}"] \arrow[dr, swap, "r_V\otimes \id_W"]  & & V\otimes (\textbf{1}\otimes W) \arrow[dl, "\id_V\otimes l_W"]  \\
& V\otimes W & 
\end{tikzcd} \quad \begin{tikzcd}
V\otimes \textbf{1} \arrow[rr, "\tau_{V, \textbf{1}}"] \arrow[dr, swap, "r_V"]  & & \textbf{1}\otimes V \arrow[dl, "l_V"]  \\
& V & 
\end{tikzcd}\]

\[\begin{tikzcd}
& (U\otimes V)\otimes W \arrow[ld, swap, "\alpha_{U, V, W}"] \arrow[dr, "\tau_{U, V}\otimes\id_W"]  &  \\
U\otimes (V\otimes W) \arrow[d, swap, "\tau_{U, V\otimes W}"] & & (V\otimes U)\otimes W \arrow[d, "\alpha_{V, U, W}"]  \\
(V\otimes W)\otimes U \arrow[dr, swap, "\alpha_{V, W, U}"] & & V\otimes (U\otimes W) \arrow[dl, "\id_V \otimes \tau_{U, W}"]\\
& V\otimes (W\otimes U) &
\end{tikzcd}\]

\end{definition}

\begin{example}
Any category \(\mathcal{C}\) with finite products has a symmetric monoidal structure (the \textit{cartesian monoidal structure}) given by the categorical product, with unit the terminal object of \(\mathcal{C}\). Dually, any category with finite coproducts has a symmetric monoidal structure (the \textit{cocartesian monoidal structure}) given by the categorical coproduct, with unit the initial object.
\end{example}

\begin{example}
The category \(\textbf{Vect}_k\) has an additional symmetric monoidal structure given by the tensor product of vector spaces, and unit object \(k\). 
\end{example}

Note the increase in complexity between the definition of a commutative monoid and that of a symmetric monoidal category; in the former case, we required that the product be strictly associative, unital, and commutative, while relaxing these conditions for the latter required us to specify extra data (the natural isomorphisms \(\alpha, l, r, \tau\)) which then had to satisfy higher coherences. Much like specifying the data for a pseudofunctor, this quickly becomes overwhelming when considering higher categories. We thus use the tools of the previous section, writing the data of a symmetric monoidal category as a pseudofunctor to \textbf{Cat}, which we can transform into an opfibration using the Grothendieck construction. 

Let \((\mathcal{C}, \otimes)\) be a symmetric monoidal category, and let \(\textbf{Fin}_*\) be the category of finite pointed sets\footnote{Replacing \(\textbf{Fin}_*\) by \(\Delta^{\text{op}}\) encodes the data of an ordinary monoidal category, since the total ordering on the finite sets removes the permutations that give rise to the natural isomorphism \(\tau\) of a symmetric monoidal category. See \autocite{groth2015short} for this formulation.}. Define a pseudofunctor 
\begin{align*}
F : \textbf{Fin}_* &\to \textbf{Cat}\\
\ang{n}&\mapsto \mathcal{C}^{\times n}
\end{align*}
Let \(f : \ang{n}\to \ang{m}\) be a morphism in \(\textbf{Fin}_*\). This induces a morphism 
\begin{align*}
f^* : (C_1, ...,C_n) &\mapsto (C_1', ..., C_m'),	
\end{align*}
where
\[C_i' = \bigotimes_{j \in f^{-1}\{i\}}C_j.\]

\begin{itemize}
%\item (How do we get a braided monoidal category? Apparently there is no base 1-category we can look at opfibrations into, because the correct formulation is with \(E_2\), which has higher homotopy groups on the mapping spaces.)
\item Correspondence of symmetric monoidal functors with morphisms of opfibrations.
\end{itemize}

Using the pseudofunctor above, along with the Grothendieck construction, we can obtain a category \(\mathcal{C}^\otimes\) and an opfibration \(p : \mathcal{C}^\otimes\to \textbf{Fin}_*\) corresponding to the pseudofunctor encoding \((\mathcal{C}, \otimes)\). Unravelling the definitions gives exactly the construction below, given in \autocite{higheralgebra}. 

Let \((\mathcal{C}, \otimes)\) be a symmetric monoidal category. We define a new category \(\mathcal{C}^\otimes\), whose objects are finite (possibly empty) sequences of objects of \(\mathcal{C}\), denoted by \([C_1, ..., C_n]\). A morphism 
\[[C_1, ..., C_n]\to [C_1', ..., C_m']\]
consists of a subset \(S \subq \{1, ..., n\}\), a map of finite sets \(\alpha : S \to \{1, ..., m\}\), and a collection of morphisms \(\{f_j : \bigotimes_{i \in \alpha^{-1}\{j\}} C_i \to C_j'\}_{1 \leq j \leq m}\) in \(\mathcal{C}\). 

For two morphisms \(f : [C_1, ..., C_n]\to [C_1', ..., C_m']\) and \(g : [C_1', ..., C_m']\to[C_1'', ..., C_l'']\), determining two subsets \(S \subq \{1, ..., n\}\) and \(T \subq \{1, ..., m\}\) and maps \(\alpha : S \to \{1, ..., m\}\), \(\beta : T \to \{1, ..., l\}\), the composition \(g \circ f\) is given by the subset \(U = \alpha^{-1}T \subq \{1, ..., n\}\), the map \(\beta \circ \alpha : U \to \{1, ..., l\}\) and the maps 
\[\left\{\bigotimes_{i \in (\beta \circ \alpha)^{-1}\{k\}} C_i \cong \bigotimes_{j \in \beta^{-1}\{k\}}\bigotimes_{i \in \alpha^{-1}\{j\}}C_i \to \bigotimes_{j \in \beta^{-1}\{k\}}C_j' \to C_k''\right\}_{1 \leq k \leq l}.\] 

For example, let 
\[f : [C_1, C_2, C_3, C_4]\to [C_1', C_2', C_3']\]
be a morphism in \(\mathcal{C}^\otimes\) consisting of the subset \(\{1, 2, 3\}\subq \{1, 2, 3, 4\}\), the map
\begin{align*}
\alpha : \{1, 2, 3\}&\to \{1, 2, 3\},\\
1 &\mapsto 1,\\
2 &\mapsto 2,\\
3 &\mapsto 3,
\end{align*}
and morphisms
\[f_1 : C_1 \to C_1', \quad f_2 : C_2 \otimes C_3 \to C_2', \quad f_3 : \textbf{1}\to C_3',\]
and let 
\[g : [C_1', C_2', C_3'] \to [C_1'', C_2'', C_3'']\]
be a morphism in \(\mathcal{C}^\otimes\) consisting of the subset \(\{1, 2, 3\}\subq \{1, 2, 3\}\), the map
\begin{align*}
\beta : \{1, 2, 3\}&\to \{1, 2, 3\},\\
1, 2, 3 &\mapsto 3,
\end{align*}
and morphisms
\[g_1 : \textbf{1} \to C_1'', \quad g_2 : \textbf{1} \to C_2'', \quad g_3 : C_1'\otimes C_2'\otimes C_3' \to C_3''.\]
Then the composition \(g \circ f \) consists of the subset \(\alpha^{-1}\{1, 2, 3\}=\{1, 2, 3\}\subq \{1, 2, 3, 4\}\), the map
\begin{align*}
\beta \circ \alpha : \{1, 2, 3\}&\to \{1, 2, 3\},\\
1, 2, 3 &\mapsto 3,
\end{align*}
and the morphisms
\[(g\circ f)_1 =g_1, \quad (g\circ f)_2 = g_2, \quad (g \circ f)_3 = g_3 \circ (f_1 \otimes f_2 \otimes f_3).\]
(really?)

(some intuition on this, tensor along the fibres, etc)

Claim: the forgetful functor
\begin{align*}
p : \quad\;\;\, \mathcal{C}^\otimes \quad\;\;\,&\to \;\;\Fin_*,\\
[C_1, ..., C_n] &\mapsto \;\;\ang{n}_*
\end{align*}
is an opfibration. (It almost tautologically is). 

\begin{proposition}[{\autocite{groth2015short}, Prop 4.26}]
If \((\mathcal{C}, \otimes)\) is a symmetric monoidal category, then the forgetful functor \(p : \mathcal{C}^\otimes \to \textbf{Fin}_*\) given above is an opfibration. Moreover, \(p\) satisfies the \textit{Segal condition}; that is, the Segal maps 
\[(\rho_!^1, ..., \rho_!^n) : \mathcal{C}^\otimes_{\ang{n}}\to \mathcal{C}^{\times n}\]
are equivalences. Conversely any Grothendieck opfibration \(p : \mathcal{C} \to \textbf{Fin}_*\) satisfying the Segal condition gives rise to a symmetric monoidal structure on \(\mathcal{C}_{\ang{1}}\).
\end{proposition}

It is worth pausing to summarise what we have achieved in the last two sections. First, we noticed that pseudofunctors \(\mathcal{C}\to \textbf{Cat}\) encode the same information as opfibrations into \(\mathcal{C}\). We also noticed that symmetric monoidal categories could be written as special pseudofunctors \(\textbf{Fin}_* \to \textbf{Cat}\), which means they are special opfibrations into \(\textbf{Fin}_*\). We looked at the corresponding construction of \(\mathcal{C}^\otimes\) (via the Grothendieck construction). Thus, symmetric monoidal categories, special pseudofunctors \(\textbf{Fin}_* \to \textbf{Cat}\), and special opfibrations into \(\textbf{Fin}_*\) all encode the same data, but the latter package allows us to `hide' the unwieldy higher coherences in the opfibration itself, which will be incredibly useful when we begin to consider symmetric monoidal \(\infty\)-categories.

We've said that symmetric monoidal categories are special opfibrations to \(\textbf{Fin}_*\). Now, what are symmetric monoidal functors? 

The definition below is basically definition 3.3 of \autocite{symmetricmonoidal}.

\begin{definition}
Let \(p : X \to \mathcal{C}\) and \(q : Y \to \mathcal{C}\) be two Grothendieck (op)fibrations. A functor \(F : X \to Y\) is a \textit{morphism of (op)fibrations} from \(p\) to \(q\) if the diagram below commutes,
\[\begin{tikzcd}
X \arrow[rd, swap, "p"] \arrow[rr, "F"]  & & Y \arrow[ld, "q"]  \\
& \mathcal{C}  & 
\end{tikzcd}\]
and \(F\) sends \(p\)-(co)cartesian morphisms to \(q\)-(co)cartesian morphisms. 
\end{definition}

\section{Generalisation to \(\infty\)-categories}

\begin{itemize}
\item If an \(\infty\)-category has finite (co)products, there is a (co)cartesian monoidal structure on \(\mathcal{C}\). And we would have hoped so, because it's definitely true for \(1\)-categories!
%\item Might be cool to try to look at \(E_k\) algebras, to resolve the earlier mystery of how to write braided monoidal categories.
\end{itemize}

We first need an \(\infty\)-categorical analogue of Grothendieck opfibrations. We start by requiring that our functor is what's known as an \textit{inner fibration}; there is no \(1\)--categorical analogue of this, since all functors between \(1\)-categories are automatically inner fibrations under the nerve functor (see \ref{ex:inner}). Think of it as a `minimum niceness condition' -- we want the fibres to be \(\infty\)-categories in much the same way as we want the fibres of ordinary functors to be categories themselves. 

\begin{definition}[{\autocite{goodfibrations}, Def 2.1}]
A functor \(p : X \to Y\) between simplicial sets is an \textit{inner fibration} if for all \(n \geq 2\), all \(0 < k < n\), and any solid arrow commutative square 
\[\begin{tikzcd}
\Lambda^n_k \arrow[r, ""] \arrow[d, hookrightarrow, swap, ""]  & X \arrow[d, "p"]  \\
\Delta^n \arrow[r, swap, ""] \arrow[ur, dashrightarrow, swap, ""]  & Y
\end{tikzcd}\]
there exists a dotted lift. 
\end{definition}

\begin{example}\label{ex:inner}
Let \(\mathcal{C}, \mathcal{D}\) be categories, and \(p : \mathcal{C} \to \mathcal{D}\) be a functor between them. Then \(N(p) : N \mathcal{C} \to N \mathcal{D}\) is an inner fibration.
\end{example}

The following proposition is stated without proof in Section 2.3 of \autocite{lurie2008higher}.

\begin{proposition}\label{pullback}
Let \(p : X \to Y\) be an inner fibration, and suppose that the diagram below is a pullback square in \textbf{sSet}.
\[\begin{tikzcd}
X' \arrow[r, "f"] \arrow[d, swap, "p'"] \arrow[dr, phantom, "\scalebox{1.3}{$\lrcorner$}" {xshift=-16pt, yshift=10pt}] & X \arrow[d, "p"]  \\
Y' \arrow[r, swap, "g"]  & Y
\end{tikzcd}\]
Then \(p'\) is also an inner fibration. 
\end{proposition}

\begin{ourproof}
Consider the (commutative) solid arrow diagram below.
\[\begin{tikzcd}
\Lambda^n_k \arrow[r, "\lambda"] \arrow[d, hookrightarrow, swap, "\iota"]  & X' \arrow[d, "p'" {yshift=-4pt}] \arrow[r, "f"] \arrow[dr, phantom, "\scalebox{1.3}{$\lrcorner$}" {xshift=-16pt, yshift=12pt}] & X \arrow[d, "p"]  \\
\Delta^n \arrow[r, swap, "\delta"] \arrow[urr, dashrightarrow, "\phi" {xshift=-8pt, yshift=-4pt}]  & Y' \arrow[r, swap, "g"] & Y
\end{tikzcd}\]
Since \(p\) is a fibration, there exists a dotted lift \(\phi\) of \(g \delta\); that is, \(p\phi = g \delta\) and \(\phi \iota = f \lambda\). Further, since the right square is a pullback diagram, there exists a unique map \(\phi' : \Delta^n \to X'\) making the diagram below commute.
\[\begin{tikzcd}
\Delta^n \arrow[rd, dashrightarrow, "\phi'"] \arrow[drr, bend left, "\phi"] \arrow[ddr, swap, bend right, "\delta"] & &  \\
 & X' \arrow[r, "f"] \arrow[d, swap, "p'"] \arrow[dr, phantom, "\scalebox{1.3}{$\lrcorner$}" {xshift=-16pt, yshift=10pt}]  & X \arrow[d, "p"] \\
 & Y' \arrow[r, swap, "g"] & Y
\end{tikzcd}\]
It remains to show that the triangle below commutes.
\[\begin{tikzcd}
\Lambda^n_k \arrow[r, "\lambda"] \arrow[d, hookrightarrow, swap, "\iota"]  & X' \\
\Delta^n \arrow[ur, dashrightarrow, swap, "\phi'"]  & 
\end{tikzcd}\]
Again, using the universal property of pullbacks, we see that there exist unique dotted maps such that the diagrams below commute.
\[\begin{tikzcd}
\Lambda^n_k \arrow[rd, dashrightarrow, ""] \arrow[drr, bend left, "f \lambda"] \arrow[ddr, swap, bend right, "\delta\iota"] & &  \\
 & X' \arrow[r, "f"] \arrow[d, swap, "p'"] \arrow[dr, phantom, "\scalebox{1.3}{$\lrcorner$}" {xshift=-16pt, yshift=10pt}] & X \arrow[d, "p"] \\
 & Y' \arrow[r, swap, "g"] & Y
\end{tikzcd} \quad \begin{tikzcd}
\Lambda^n_k \arrow[rd, dashrightarrow, ""] \arrow[drr, bend left, "f\phi'\iota"] \arrow[ddr, swap, bend right, "\delta\iota"] & &  \\
 & X' \arrow[r, "f"] \arrow[d, swap, "p'"] \arrow[dr, phantom, "\scalebox{1.3}{$\lrcorner$}" {xshift=-16pt, yshift=10pt}] & X \arrow[d, "p"] \\
 & Y' \arrow[r, swap, "g"] & Y
\end{tikzcd}\]
The maps \(\lambda\) and \(\phi'\iota\) make the left and right diagrams commute respectively. Further, we note that \(f\phi' = \phi\) (by the second diagram) and \(\phi \iota =f \lambda\) (since \(p\) is an inner fibration), so \(f\phi'\iota=f \lambda\). Therefore, the above two diagrams are identical. Thus, by the uniqueness property of pullbacks, \(\lambda = \phi'\iota\). \done
\end{ourproof}

%\begin{corollary}
%Let \(p : X \to Y\) be an inner fibration. Then each fibre \(X \times_{Y}\{y\}\) is an \(\infty\)-category.
%\end{corollary}

\begin{example}[{\autocite{goodfibrations}, Ex 2.2}]
Let \(p : X \to \Delta^0\) be the canonical map, and suppose we have the diagram below, such that the outer square commutes. 
\[\begin{tikzcd}
\Lambda^n_k \arrow[r, ""] \arrow[d, hookrightarrow, swap, ""]  & X \arrow[d, "p"]  \\
\Delta^n \arrow[r, swap, ""] \arrow[ur, dashrightarrow, swap, ""]  & \Delta^0
\end{tikzcd}\]
The lower triangle commutes automatically, so the statement that \(p\) is an inner fibration is equivalent to the statement that for all \(n \geq 2\), all \(0 < k < n\), and any map \(\Lambda^n_k \to X\), there exists a dotted lift.
\[\begin{tikzcd}
\Lambda^n_k \arrow[r, ""] \arrow[d, hookrightarrow, swap, ""]  & X  \\
\Delta^n \arrow[ur, dashrightarrow, swap, ""]  & 
\end{tikzcd}\]
That is, \(X\) is an \(\infty\)-category.

Now, combining the above argument with \ref{pullback}, we see that for any inner fibration \(p : X \to Y\), each fibre \(X\times_Y \{y\}\) is an \(\infty\)-category. 
\end{example}

\begin{definition}[{\autocite{goodfibrations}, Def 3.1}]
Let \(p : X \to Y\) be an inner fibration. An edge \(f : \Delta^1 \to X\) of \(X\) is \(p\)-\textit{cocartesian} if for all \(n \geq 2\), any extension 
\[\begin{tikzcd}
\Delta^{\{0, 1\}} \arrow[r, "f"] \arrow[d, hookrightarrow, swap, ""]  & X \\
\Lambda^n_0 \arrow[ur, swap, "F"]  & 
\end{tikzcd}\]
and any solid arrow commutative diagram
\[\begin{tikzcd}
\Lambda^n_0 \arrow[r, "F"] \arrow[d, hookrightarrow, swap, ""]  & X \arrow[d, "p"]  \\
\Delta^n \arrow[r, swap, ""] \arrow[ur, dashrightarrow, ""]  & Y
\end{tikzcd}\]
a dotted lift exists. 
\end{definition}

\begin{definition}
Let \(p : X \to Y\) be an inner fibration. Then \(p\) is a cocartesian fibration if for any edge \(\phi : y \to y'\) in \(Y_1\), and for every \(x \in X_0\) lying over \(y\), there exists a \(p\)-cocartesian edge \(f : x \to x'\) of \(X\) lying over \(\phi\). 
\end{definition}

The following proposition (stated as a remark in \autocite{lurie2008higher}) tells us that the above definition is a reasonable generalisation of \ref{opfibration}. 

\begin{proposition}[{\autocite{lurie2008higher}, Rmk 2.4.2.2}]
Let \(\mathcal{C}\), \(\mathcal{D}\) be categories, and let \(p : \mathcal{C} \to \mathcal{D}\) be a functor between them. Then \(p\) is a Grothendieck opfibration if and only if the induced map \(N(p) : N \mathcal{C} \to N\mathcal{D}\) is a cocartesian fibration of simplicial sets.
\end{proposition}

\begin{ourproof}
Let \(f : d \to d'\) be a morphism of \(\mathcal{D}\), and let \(c\) lie over \(d\). 

Suppose \(p\) is a Grothendieck opfibration, let \(F : \Lambda^n_0 \to N \mathcal{C}\) be an extension of \(f\), and let 
\[\begin{tikzcd}
\Lambda^n_0 \arrow[r, "F"] \arrow[d, hookrightarrow, swap, ""]  & N\mathcal{C} \arrow[d, "N(p)"]  \\
\Delta^n \arrow[r, swap, ""]  & N\mathcal{D}
\end{tikzcd}\]
be a commutative diagram. If \(n=2\), it follows immediately from the fact that \(p\) is an opfibration that a dotted lift exists. Further, if \(n>3\), there is nothing to check, since an \(n\)-simplex in a category commutes if and only if all of its triangles commute, which is guaranteed for any extension \(F : \Lambda^n_0 \to N \mathcal{C}\). We thus prove the proposition for \(n=3\).

Suppose we have an extension \(F : \Lambda^3_0 \to N \mathcal{C}\) of \(f\); that is, a tetrahedron 
\[\begin{tikzcd}
& c \arrow[rr, "\chi"] \arrow[ddl, swap, "\phi"]  & & c''' \\
& & & \\
c' \arrow[rr, swap, "\gamma"] \arrow[rrruu, swap, "\gamma'" {xshift=8pt, yshift=6pt}] & & c'' \arrow[ruu, swap, "\gamma''"] \arrow[from=luu, swap, crossing over, "\psi" {xshift=-4pt, yshift=8pt}] & 
\end{tikzcd}\]
such that all faces containing the vertex \(c\) commute. Let
\[\begin{tikzcd}
& d \arrow[rr, "p(\chi)"] \arrow[ddl, swap, "f"]  & & p(c''') \\
& & & \\
d' \arrow[rr, swap, "p(\gamma)"] \arrow[rrruu, swap, "p(\gamma')" {xshift=8pt, yshift=6pt}] & & p(c'') \arrow[ruu, swap, "p(\gamma'')"] \arrow[from=luu, swap, crossing over, "p(\psi)" {xshift=-4pt, yshift=8pt}] & 
\end{tikzcd}\]
be a commutative tetrahedron in \(\mathcal{D}\). We claim that the tetrahedron in \(\mathcal{C}\) commutes. First, note that \(\gamma'' \circ \gamma\) is a lift of \(p(\gamma')\), since \(p(\gamma')=p(\gamma'')\circ p(\gamma)=p(\gamma'' \circ \gamma)\). Further,
\begin{align*}
(\gamma \circ \gamma'') \circ \phi&=\gamma'' \circ \psi\\
&= \chi.
\end{align*}
Thus, by the uniqueness in the universal property of \(\phi\), we have that \(\gamma'=\gamma'' \circ \gamma\), as required.

Now, suppose \(N(p)\) is a cocartesian fibration. Then there exists a lift \(\phi : c \to c'\) of \(f\), and, in particular, for any diagram
\[\begin{tikzcd}
& c \arrow[dl, swap, "\phi"] \arrow[dr, "\psi"]  & \\
c'  & & c''
\end{tikzcd}\]
in \(\mathcal{C}\), and any commutative diagram
\[\begin{tikzcd}
& d \arrow[dl, swap, "f"] \arrow[dr, "p(\psi)"]  & \\
d' \arrow[rr, swap, "g"] & & p(c'')
\end{tikzcd}\]
in \(\mathcal{D}\), there exists a map \(\gamma : c'\to c''\) such that \(\gamma\) lies over \(g\) and \(\gamma \circ \phi = \psi\). It remains to show that \(\gamma\) is unique. 

Suppose that there were two maps \(\gamma_1, \gamma_2 : c' \to c''\) lying over \(g\) and satisfying \(\gamma_1\circ\phi = \gamma_2 \circ \phi = \psi\). Then we would have a tetrahedron 

\[\begin{tikzcd}
& c \arrow[rr, "\psi"] \arrow[ddl, swap, "\phi"]  & & c'' \\
& & & \\
c' \arrow[rr, swap, "\gamma_1"] \arrow[rrruu, swap, "\gamma_2" {xshift=8pt, yshift=5pt}] & & c'' \arrow[ruu, swap, "\id"] \arrow[from=luu, swap, crossing over, "\psi" {xshift=-4pt, yshift=8pt}] & 
\end{tikzcd}\]
where all faces containing the vertex \(c\) commute. The image of this tetrahedron under \(p\) commutes in \(\mathcal{D}\), so the original tetrahedron must commute in \(\mathcal{C}\); that is, \(\gamma_1=\gamma_2\).\done

\begin{definition}[{\autocite{lurie2008higher}, Def 2.0.0.7}]
A \textit{symmetric monoidal \(\infty\)-category} is a cocartesian fibration of simplicial sets \(p : X^\otimes \to N(\textbf{Fin}_*)\) such that for each \(n \geq 0\), the maps 
\[\{\rho^i : \ang{n}\to \ang{1}\}_{1 \leq i \leq n}\] 
induce functors \(\rho^i_! : X^\otimes_{\ang{n}}\to X^\otimes_{\ang{1}}\) which determine an equivalence \(X^\otimes_{\ang{n}}\simeq \left(X^\otimes_{\ang{1}}\right)^n\). 
\end{definition}

\begin{example}
Let \((\mathcal{C}, \otimes)\) be a symmetric monoidal category. Then \(p : N(\mathcal{C}^\otimes) \to N(\textbf{Fin}_*)\) is a symmetric monoidal \(\infty\)-category. 
\end{example}

\section{A nontrivial example}

Throughout this section, \(\mathcal{A}\) is an abelian category, and \(\mathcal{A}_{\text{proj}}\) is the full subcategory of \(\mathcal{A}\) spanned by the projective objects. 

\begin{definition}[{\autocite{higheralgebra}, Def 1.2.3.1}]
A \textit{chain complex} with values in \(\mathcal{A}\) is a composable sequence of morphisms 
\[\cdots \to A_2 \xrightarrow{d(2)} A_1 \xrightarrow{d(1)}A_0 \xrightarrow{d(0)} A_{-1}\to \cdots\]
in \(\mathcal{A}\) such that \(d(n-1)\circ d(n)=0\) for all \(n \in \inte\). The collection of chain complexes with values in \(\mathcal{A}\) is an additive category, \(\Ch(\mathcal{A})\). 
\end{definition}

\begin{definition}[\autocite{higheralgebra}, Not 1.3.2.6]
\(\Ch^-(\mathcal{A})\) is the full subcategory of \(\Ch(\mathcal{A})\) spanned by those chain complexes \(M_*\) such that \(M_n \simeq 0\) for \(n << 0\). 
\end{definition}

\begin{definition}[{\autocite{higheralgebra}, Def 1.3.2.7}]
Suppose \(\mathcal{A}\) has enough projective objects. We let \(\mathcal{D}^-(\mathcal{A})\) denote the \(\infty\)-category \(N_{\text{dg}}(\Ch^-(\mathcal{A}_{\text{proj}}))\). We refer to \(\mathcal{D}^-(\mathcal{A})\) as the \textit{derived \(\infty\)-category of \(\mathcal{A}\)}.  
\end{definition}

%Let 
%\[\begin{tikzcd}
%& c \arrow[dl, swap, "\phi"] \arrow[dr, "\psi"]  & \\
%c'  & & c''
%\end{tikzcd}\quad \begin{tikzcd}
%& c \arrow[dl, swap, "\phi"] \arrow[dr, "\chi"]  & \\
%c'  & & c'''
%\end{tikzcd}\]
%be diagrams in \(\mathcal{C}\), and let
%\[\begin{tikzcd}
%& d \arrow[dl, swap, "f"] \arrow[dr, "p(\psi)"]  & \\
%d' \arrow[rr, swap, "g"] & & p(c'')
%\end{tikzcd}\quad \begin{tikzcd}
%& d \arrow[dl, swap, "f"] \arrow[dr, "p(\chi)"]  & \\
%d' \arrow[rr, swap, "g'"] & & p(c''')
%\end{tikzcd}\]
%be commutative diagrams in \(\mathcal{D}\). Then there exist maps \(\gamma : c' \to c''\) and \(\gamma' : c' \to c'''\) lying over \(g\) and \(g'\) respectively, and making the diagrams in \(\mathcal{C}\) commute. Suppose there were two maps \(\gamma'_1 \gamma'_2 : c' \to c'''\) lying over \(g'\) and making the right diagram in \(\mathcal{C}\) commute. 

%Again, since \(N(p)\) is a cocartesian fibration, for any tetrahedron
%\[\begin{tikzcd}
%& c \arrow[rr, "\chi"] \arrow[ddl, swap, "\phi"]  & & c''' \\
%& & & \\
%c' \arrow[rr, swap, ""] \arrow[rrruu, swap, ""] & & c'' \arrow[ruu, swap, ""] \arrow[from=luu, swap, crossing over, "\psi"] & 
%\end{tikzcd}\]
%where all faces containing the vertex \(c\) commute, if image commutes in \(\mathcal{D}\), then the original diagram must commute. 
\end{ourproof}

%This is a test to check pushing to GitHub is working as expected.

%\tableofcontents

%\pagebreak

\section{Miscellaneous stupid notes}

\subsection{Observations}

(Stupid note to self, very obvious but I forget it every now and again):
\begin{itemize}
\item If \(X : \Delta^{\text{op}}\to \textbf{Set}\) is a simplicial set, and \(\Delta^0 : \Delta^{\text{op}}\to \textbf{Set}:= \Hom(-, [0])\), then a map \(F : X \to \Delta^0\) is a natural transformation \((F_n : X_n \to *)_{n \in \nat_0}\). That is, such a natural transformation is a family of maps down to a point. In other words, there's only really one natural transformation, so we really *can* view \(\Delta^0\) as a point. 
\item If \(Y\) is a simplicial set, and \(y \in Y_0\) is a vertex of \(Y\), we can view \(\{y\}\) as a copy of \(\Delta^0\). Why is this? We can view \(\{y\}\) as the constant simplicial set, sending everything to \(y\). Then a natural isomorphism \(\Delta^0 \cong \{y\}\) is a collection of isomorphisms \((* \to *)\), of which there is exactly one. Why is it natural? Well, there's only one map from a one-point set to another one-point set, so the square always commutes. 
\end{itemize}

Let \(S \in \textbf{Set}\). We define the constant simplicial set 
\[\begin{tikzcd}
\overline{S} : \Delta^{\text{op}} \arrow[r, ""]  & \textbf{Set} \\
\left[n\right] \arrow[r, mapsto, swap, ""] \arrow[d, swap, "f"] & S\\
\left[m\right] \arrow[r, mapsto, swap, ""] & S \arrow[u, swap, "\id"]
\end{tikzcd}\]

It's a Kan complex. Why? Well, when you consider \(S\) as a discrete category, and take the nerve of it, you get \(\overline{S}\). You can then either just see than it's a Kan complex (fill the horns with identities) or use the fact that the nerve of a groupoid is a Kan complex. It's surely in \autocite{lurie2008higher} or \autocite{kerodon} somewhere. 

Here are the original definitions of the Grothendieck equivalence (before I dualised them, just in case I did it wrong). 

Let \(F : \mathcal{C}^{\text{op}} \to \textbf{Cat}\) be a pseudofunctor. Define the category \(X\) as follows: the objects of \(X\) are pairs \((c, x)\), with \(c \in \mathcal{C}\), \(x \in F(c)\). A map \((c, x)\to (d, y)\) is a pair \((f, u)\), where \(f : c \to d\) is a morphism in \(\mathcal{C}\), and \(u : x\to (Ff)(y)\) is a morphism in \(F(c)\). For an object \((c, x) \in X\), the identity morphism is given by
\[(\id_c, F^0_x : x \to F(\id_c)(x)),\]
where \(F^0\) is the natural isomorphism \(\id_{F(c)} \cong F(\id_c)\). Further, given two maps \((f, u) : (c, x) \to (d, y)\) and \((g, v) : (d, y)\to (e, z)\), their composition \((g,v)\circ (f, u)\) is given by \(g \circ f\), together with the map
\[x \xrightarrow{u} (Ff)(y)\xrightarrow{(Ff)(v)} (Ff \circ Fg)(z) \xrightarrow{\left(F^2_{f,g}\right)_z} (F(g\circ f))(z),\]

where \(F^2_{f, g}\) is the natural isomorphism \(Ff \circ Fg \cong F(g \circ f)\) (recall that the domain of \(F\) is \(\mathcal{C}^{\text{op}}\), while \(f, g\) are morphisms in \(\mathcal{C}\)). One can show that the forgetful functor \(X \to \mathcal{C}\) is a fibration over \(\mathcal{C}\). 

Now, let \(p : X\to \mathcal{C}\) be a fibration over \(\mathcal{C}\). Define a pseudofunctor 
\begin{align*}
\mathcal{C}^{\text{op}}&\to \textbf{Cat}\\
c &\mapsto p^{-1}\{c\}.
\end{align*}

For a map \(f : c \to d\) in \(\mathcal{C}\), we define a functor 
\begin{align*}
f^* : p^{-1}\{d\}&\to p^{-1}\{c\},
\end{align*}
which sends \(x \in p^{-1}\{d\}\) to a the source \(x'\) of a cartesian edge \(\phi_x : x' \to x\) lying over \(f\). Now, let \(f^*(y)=y'\), with \(\phi_y : y' \to y\), and let \(g : x \to y\) be a morphism in \(p^{-1}\{d\}\) (that is, a morphism in \(X\) lying over \(\id_c\)). Then, since \(\phi_y : y' \to y\) is a cartesian edge, there is a unique lift \(\overline g : x' \to y'\) of \(\id_c\) making the square on the left commute, as shown in the diagram on the right. 

\[\begin{tikzcd}
x' \arrow[r, "\phi_x"] \arrow[d, swap, dashrightarrow, "\exists!\overline g"]  & x \arrow[d, "g"]  \\
y' \arrow[r, swap, "\phi_y"]  & y
\end{tikzcd}\quad\quad\quad \begin{tikzcd}
 & & x' \arrow[dddd, mapsto, ""] \arrow[dll, dashrightarrow, swap, "\exists !\overline g"] \arrow[dl, "g \circ \phi_x"] \\
y' \arrow[r, swap, "\phi_y"] \arrow[dd, mapsto, swap, ""] & y \arrow[dd, mapsto, swap, ""]  & \\
& & \\
c \arrow[r, "f"] & d  & \\
 & & c \arrow[llu, "\id_c"] \arrow[lu, swap, "f"] 
\end{tikzcd}\]

We thus define \(f^*g=\overline{g}\) in \(p^{-1}\{c\}\). It can be shown that these data assemble into a pseudofunctor \(\mathcal{C}^{\text{op}}\to \textbf{Cat}\).

\subsection{Questions}

Questions:
\begin{itemize}
\item ...what *is* \(\textbf{Grpd}_\infty\)?
\end{itemize}

\subsection{Equivalent definitions}

\(\textbf{Grpd}_\infty\) 

An algebraic category

An equivalence of \(\infty\)-categories


\printbibliography

\end{document}