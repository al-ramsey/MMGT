\documentclass{MetricNotes2023}

% Language setting
% Replace `english' with e.g. `spanish' to change the document language


% Set page size and margins
% Replace `letterpaper' with `a4paper' for UK/EU standard size
\usepackage[a4paper,top=2cm,bottom=2cm,left=2cm,right=2cm,marginparwidth=1.75cm]{geometry}

% Useful packages
\usepackage{graphicx}
\graphicspath{ {./images/} }
\usepackage{float}


%inserting a figure:

%\begin{figure}[h]
%\centering
%\includegraphics[width=0.5\textwidth]{pfigure_1}
%\caption{The compound pendulum}
%\end{figure}
\setcounter{MaxMatrixCols}{20}
\usepackage{amsmath}
\usepackage{enumitem}
\usepackage{amssymb}
\usepackage{amsthm}
\usepackage{realhats} %for hats
\usepackage[colorlinks=true, allcolors=blue]{hyperref}
%\usepackage{hyperref}
\usepackage{mathtools}
\usepackage{parskip}
%\usepackage{mbboard}
%\usepackage[mtpbbi]{mtpro2}
\usepackage{mathrsfs}


% knots stuff

\usepackage{tikz}
\usetikzlibrary{cd}
\usetikzlibrary{shapes,snakes}
\usepackage{listings}

%\graphicspath{ {figs/} }

\usepackage{cleveref}
\newcommand{\notimplies}{%
  \mathrel{{\ooalign{\hidewidth$\not\phantom{=}$\hidewidth\cr$\implies$}}}}

%keyword: edit

\def\summing{\ensuremath\sum^{\infty}_{n=1}}
\def\summingo{\ensuremath\sum^{\infty}_{n=0}}
\def\bb{\ensuremath\mathbb}
\def\la{\ensuremath\langle}
\def\ra{\ensuremath\rangle}
\def\notto{\ensuremath\nrightarrow}
\def\subq{\ensuremath\subseteq}

\def\met{\ensuremath(X, d)}
\def\metx{\ensuremath(X, d_X)}
\def\mety{\ensuremath(Y, d_Y)}
\def\topox{\ensuremath(X, \mathcal{T}_X)}
\def\topoy{\ensuremath(Y, \mathcal{T}_Y)}

\def\comp{\ensuremath\mathbb{C}}
\def\real{\ensuremath\mathbb{R}}
\def\rat{\ensuremath\mathbb{Q}}
\def\inte{\ensuremath\mathbb{Z}}
\def\nat{\ensuremath\mathbb{N}}
\def\topo{\ensuremath\mathcal{T}}

\def\sequiv{\ensuremath\stackrel{\text{S}}{\sim}}
\def\rr{\ensuremath\mathcal{R}}
\def\com{\ensuremath\Delta}
\def\id{\ensuremath\text{id}}
\def\hopf{\ensuremath\mathcal{H}}
\def\psip{\ensuremath\psi_{+}}
\def\psin{\ensuremath\psi_{-}}
\def\psipm{\ensuremath\psi_{\pm}}
\DeclareMathOperator{\trace}{trace}
\DeclareMathOperator{\tr}{tr}
\DeclareMathOperator{\str}{str}
\DeclareMathOperator{\End}{End}
\DeclareMathOperator{\Hom}{Hom}
\DeclareMathOperator{\lk}{lk}
\DeclareMathOperator{\ob}{ob}
\DeclareMathOperator{\Fin}{Fin}

\def\done{\begin{flushright}\(\qed\)\end{flushright}}

\def\mlip{\ensuremath\abs f_\text{Lip}}


%\DeclareSymbolFont{bbold}{U}{bbold}{m}{n}
%\DeclareSymbolFontAlphabet{\mathbbold}{bbold}

% \limits\sum or \sum\limits (one of them) will put the text under the sum
% [upquote=true] on \begin{lstlisting} will stop backticks being interpreted as quotes

\usepackage[backend=bibtex]{biblatex}
\usepackage{csquotes}
\addbibresource{References.bib}

%\author{Ali Ramsey}
\title{Symmetric Monoidal \(\infty\)-Categories}
\date{\vspace{-5ex}}

\counterwithin{figure}{section}


\begin{document}
\maketitle
%\input{titlepage.tex}

\DeclarePairedDelimiter{\norm}{\lVert}{\rVert} 
\DeclarePairedDelimiter{\abs}{\lvert}{\rvert} 
\DeclarePairedDelimiter{\ang}{\langle}{\rangle} 

(Assuming I don't change topics, this is a rough idea of what I want to talk about).

Points to cover:
\begin{itemize}
\item Grothendieck op-fibrations correspond to pseudofunctors to \textbf{Cat}.
\item (Usual) definition of symmetric monoidal category and translation into the language of op-fibrations into \(\text{Fin}_*\), running example of \(\textbf{Vect}_k\) with \(\otimes\) or \(\times\). 
\item Possibly mention swapping out \(\textbf{Fin}_*\) for \(\Delta^{\text{op}}\) gives a monoidal category rather than a symmetric monoidal category (how do we get a braided monoidal category?)
\item Correspondence of symmetric monoidal functors with morphisms of op-fibrations.
\item Translation of the above into \(\infty\)-categorical language.
\item Note: we will call \((\infty, 1)\)-categories \(\infty\)-categories. Sorry Tom, it's just convenient. 
\item Some examples (nerve of an ordinary symmetric monoidal category, currently trying to find more examples -- many people talk about \textbf{Sp}, but it seems like I'd need a lot of background to understand this).
\item Possibly generalisation to \(\infty\)-operads, depending on how much the above comes to or if I find anything fun to do with symmetric monoidal \(\infty\)-categories. 
\item Main references: \autocite{higheralgebra}, \autocite{lurie2008higher}, \autocite{symmetricmonoidal}, \autocite{goodfibrations}, \autocite{groth2015short}.
\end{itemize}

\section{Pseudofunctors, opfibrations, and symmetric monoidal categories}

\subsection{Pseudofunctors to \textbf{Cat} are Grothendieck opfibrations}

The below gives an example to complement \autocite{symmetricmonoidal}.

Example of a pseudofunctor to \textbf{Cat}: Let \(\mathcal{C}\) be a category with pullbacks. Recall that for a map \(f : C \to D\) in \(\mathcal{C}\), we define a pullback functor 
\begin{align*}
f^* : \;\;\;\mathcal{C}_{/D}\;\;\;&\to\;\;\; \mathcal{C}_{/C},\\
(h : X \to D) &\mapsto (f^*h : P \to C),
\end{align*}
 where we have formed a pullback 
\[\begin{tikzcd}
P \arrow[r, "h^*f"] \arrow[d, swap, "f^*h"]  & X \arrow[d, "h"]  \\
C \arrow[r, swap, "f"]  & D
\end{tikzcd}\]
in \(\mathscr{C}\). For any map 
\[\begin{tikzcd}
X \arrow[rr, "\phi"] \arrow[dr, swap, "h"]  & & X' \arrow[dl, "h'"]  \\
& D  & 
\end{tikzcd}\]
from \(h\) to \(h'\) in \(\mathcal{C}_{/D}\), we define \(f^*\phi\) to be the unique map making the diagram below commute.
\[\begin{tikzcd}
X \arrow[rr, "\phi"] \arrow[dr, swap, "h"]  & & X' \arrow[dl, "h'"]  \\
& D  & \\
& & & \\
P \arrow[rr, dashrightarrow, "f^*\phi" {xshift=-10pt}] \arrow[uuu, "h^*f"] \arrow[dr, swap, "f^*h"] & & P' \arrow[uuu, swap, "h'^*f"] \arrow[dl, "f^*h'"]\\
& C \arrow[uuu, swap, "f" {yshift=8pt}] & 
\end{tikzcd}\]
Now, we may wish to define a functor 
\begin{align*}
F : \mathcal{C}^{\text{op}} &\to \textbf{Cat},\\
C &\mapsto \mathcal{C}_{/C},
\end{align*}
which sends a map \(f : C \to D\) in \(\mathcal{C}\) to a pullback functor \(f^* : \mathcal{C}_{/D}\to \mathcal{C}_{/C}\). However a problem arises when we check that \(F\) respects composition: suppose \(f : C \to D\), \(g : D \to E\) are maps in \(\mathcal{C}\). Then 
\[F(g \circ f)(h : X \to E)=(g\circ f)^*h : P \to C,\] corresponding to the pullback
\[\begin{tikzcd}
P \arrow[r, "h^*(g\circ f)"] \arrow[d, swap, "(g \circ f)^*h"]  & X \arrow[d, "h"]  \\
C \arrow[r, swap, "g\circ f"]  & E
\end{tikzcd}\]
in \(\mathcal{C}\). On the other hand, 
\[(F(g)\circ F(f))(h : X \to E)=f^*(g^*h) : P'' \to C,\] which corresponds to the diagram below.
\[\begin{tikzcd}
P'' \arrow[r, "(g^*h)^*f"] \arrow[d, swap, "f^*(g^*h)"]  & P' \arrow[r, "h^*g"] \arrow[d, "g^*h"] & X \arrow[d, "h"]  \\
C \arrow[r, swap, "f"]  & D \arrow[r, swap, "g"] & E
\end{tikzcd}\]
The outer square is indeed a pullback square, since the inner two squares are, so we have a unique isomorphism \(P \cong P''\). However, we do not in general have equality. This is because pullbacks are only unique up to unique isomorphism, and in defining a pullback functor we made arbitrary (and not necessarily compatible) choices of \(P, P'\) and \(P''\). Thus, we have not defined a functor \(\mathcal{C}^{\text{op}}\to \textbf{Cat}\), rather, we have defined what is known as a \textit{pseudofunctor}. 

(definition of a pseudofunctor)

The above example is one way in which pseudofunctors into \textbf{Cat} naturally arise; another common example is the pseudofunctor
\begin{align*}
\textbf{CRing}&\to \textbf{Cat}\\
R\; &\mapsto R \textbf{-Mod},
\end{align*}
which sends a ring homomorphism \(\phi : R \to S\) to the functor \(-\otimes_R S : R \textbf{-Mod}\to S \textbf{-Mod}\) (extension of scalars). However, to give the data of a pseudofunctor \(F : \mathcal{C} \to \textbf{Cat}\), we must specify not only the functions \(\ob(\mathcal{C})\to \ob(\textbf{Cat})\) and \(\Hom_\mathcal{C}(X, Y)\to \Hom_\textbf{Cat}(F(X), F(Y))\) for each \(X, Y \in \mathcal{C}\), but also natural isomorphisms
\[F(\id_X) \cong \id_{F(X)}, \quad F(g \circ f) \cong F(g) \circ F(f).\]

That's a pain, let's use Grothendieck opfibrations instead. 

The definitions below are from \autocite{goodfibrations}.

\begin{definition}
Let \(p : X \to \mathcal{C}\) be a functor, and let \(f : c \to d\) be a morphism in \(\mathcal{C}\). A morphism \(\Phi\) in \(X\) lying over \(f\) is \textit{\(p\)-cocartesian} if for any other morphism \(\Psi\) of \(X\) lying over \(f\), there exists a morphism \(I\) of \(X\) such that \(p(I)=\id_d\) and \(\Psi = I \circ \Phi\). 
\end{definition}

\begin{definition}
Let \(p : X \to \mathcal{C}\) be a functor. Then \(p\) is a \textit{Grothendieck opfibration} if for any morphism of \(\mathcal{C}\) and any lift of its source, there is a \(p\)-cocartesian morphism with that course lying over it.
\end{definition}

\subsection{Symmetric monoidal categories are special pseudofunctors to \textbf{Cat}}

\subsection{...which are special opfibrations}

The construction below is in \autocite{higheralgebra}. 

Let \((\mathcal{C}, \otimes)\) be a symmetric monoidal category. We define a new category \(\mathcal{C}^\otimes\), whose objects are finite (possibly empty) sequences of objects of \(\mathcal{C}\), denoted by \([C_1, ..., C_n]\). A morphism 
\[[C_1, ..., C_n]\to [C_1', ..., C_m']\]
consists of a subset \(S \subq \{1, ..., n\}\), a map of finite sets \(\alpha : S \to \{1, ..., m\}\), and a collection of morphisms \(\{f_j : \bigotimes_{i \in \alpha^{-1}\{j\}} C_i \to C_j'\}_{1 \leq j \leq m}\) in \(\mathcal{C}\). 

For two morphisms \(f : [C_1, ..., C_n]\to [C_1', ..., C_m']\) and \(g : [C_1', ..., C_m']\to[C_1'', ..., C_l'']\), determining two subsets \(S \subq \{1, ..., n\}\) and \(T \subq \{1, ..., m\}\) and maps \(\alpha : S \to \{1, ..., m\}\), \(\beta : T \to \{1, ..., l\}\), the composition \(g \circ f\) is given by the subset \(U = \alpha^{-1}T \subq \{1, ..., n\}\), the map \(\beta \circ \alpha : U \to \{1, ..., l\}\) and the maps 
\[\left\{\bigotimes_{i \in (\beta \circ \alpha)^{-1}\{k\}} C_i \cong \bigotimes_{j \in \beta^{-1}\{k\}}\bigotimes_{i \in \alpha^{-1}\{j\}}C_i \to \bigotimes_{j \in \beta^{-1}\{k\}}C_j' \to C_k''\right\}_{1 \leq k \leq l}.\] 

For example, let 
\[f : [C_1, C_2, C_3, C_4]\to [C_1', C_2', C_3']\]
be a morphism in \(\mathcal{C}^\otimes\) consisting of the subset \(\{1, 2, 3\}\subq \{1, 2, 3, 4\}\), the map
\begin{align*}
\alpha : \{1, 2, 3\}&\to \{1, 2, 3\},\\
1 &\mapsto 1,\\
2 &\mapsto 2,\\
3 &\mapsto 3,
\end{align*}
and morphisms
\[f_1 : C_1 \to C_1', \quad f_2 : C_2 \otimes C_3 \to C_2', \quad f_3 : \textbf{1}\to C_3',\]
and let 
\[g : [C_1', C_2', C_3'] \to [C_1'', C_2'', C_3'']\]
be a morphism in \(\mathcal{C}^\otimes\) consisting of the subset \(\{1, 2, 3\}\subq \{1, 2, 3\}\), the map
\begin{align*}
\beta : \{1, 2, 3\}&\to \{1, 2, 3\},\\
1, 2, 3 &\mapsto 3,
\end{align*}
and morphisms
\[g_1 : \textbf{1} \to C_1'', \quad g_2 : \textbf{1} \to C_2'', \quad g_3 : C_1'\otimes C_2'\otimes C_3' \to C_3''.\]
Then the composition \(g \circ f \) consists of the subset \(\alpha^{-1}\{1, 2, 3\}=\{1, 2, 3\}\subq \{1, 2, 3, 4\}\), the map
\begin{align*}
\beta \circ \alpha : \{1, 2, 3\}&\to \{1, 2, 3\},\\
1, 2, 3 &\mapsto 3,
\end{align*}
and the morphisms
\[(g\circ f)_1 =g_1, \quad (g\circ f)_2 = g_2, \quad (g \circ f)_3 = g_3 \circ (f_1 \otimes f_2 \otimes f_3).\]
(really?)

(some intuition on this, tensor along the fibres, etc)

Claim: the forgetful functor
\begin{align*}
p : \quad\;\mathcal{C}^\otimes \quad\;&\to \;\;\Fin_*,\\
[C_1, ..., C_n] &\mapsto \;\ang{n}_*
\end{align*}
is an opfibration. (It almost tautologically is). 

\subsection{Generalisation to \(\infty\)-categories}

\begin{definition}[{\autocite{goodfibrations}, Def 2.1}]
A functor \(p : X \to Y\) between \(\infty\)-categories is an \textit{inner fibration} if for all \(n \geq 2\), all \(0 < k < n\), and any solid arrow commutative square 
\[\begin{tikzcd}
\Lambda^n_k \arrow[r, ""] \arrow[d, hookrightarrow, swap, ""]  & X \arrow[d, "p"]  \\
\Delta^n \arrow[r, swap, ""] \arrow[ur, dashrightarrow, swap, ""]  & Y
\end{tikzcd}\]
there exists a dotted lift. 
\end{definition}

(Stupid note to self, very obvious but I forget it every now and again):
\begin{itemize}
\item If \(X : \Delta^{\text{op}}\to \textbf{Set}\) is a simplicial set, and \(\Delta^0 : \Delta^{\text{op}}\to \textbf{Set}:= \Hom(-, [0])\), then a map \(F : X \to \Delta^0\) is a natural transformation \((F_n : X_n \to *)_{n \in \nat_0}\). That is, such a natural transformation is a family of maps down to a point. In other words, there's only really one natural transformation, so we really *can* view \(\Delta^0\) as a point. 
\end{itemize}

\begin{example}[{\autocite{goodfibrations}, Ex 2.2}]
Let \(p : X \to \Delta^0\) be the canonical map, and suppose we have the diagram below, such that the outer square commutes. 
\[\begin{tikzcd}
\Lambda^n_k \arrow[r, ""] \arrow[d, hookrightarrow, swap, ""]  & X \arrow[d, "p"]  \\
\Delta^n \arrow[r, swap, ""] \arrow[ur, dashrightarrow, swap, "\phi"]  & \Delta^0
\end{tikzcd}\]
The lower triangle commutes automatically, so the statement that \(p\) is an inner fibration is equivalent to the statement that for all \(n \geq 2\), all \(0 < k < n\), and any map \(\Lambda^n_k \to X\), there exists a dotted lift.
\[\begin{tikzcd}
\Lambda^n_k \arrow[r, ""] \arrow[d, hookrightarrow, swap, ""]  & X  \\
\Delta^n \arrow[ur, dashrightarrow, swap, ""]  & 
\end{tikzcd}\]
That is, \(X\) is an \(\infty\)-category.

Now, since any element \(y \in Y_0\) can be viewed as a copy of \(\Delta^0\), the above discussion tells us that the fibre \(p^{-1}\{y\}\) is an \(\infty\)-category.
\end{example}

\begin{definition}[{\autocite{goodfibrations}, Def 3.1}]
Let \(p : X \to Y\) be an inner fibration. An edge \(f : \Delta^1 \to X\) of \(X\) is \(p\)-\textit{cocartesian} if for all \(n \geq 2\), any extension 
\[\begin{tikzcd}
\Delta^{\{0, 1\}} \arrow[r, "f"] \arrow[d, hookrightarrow, swap, ""]  & X \\
\Lambda^n_0 \arrow[ur, swap, "F"]  & 
\end{tikzcd}\]
and any solid arrow commutative diagram
\[\begin{tikzcd}
\Lambda^n_0 \arrow[r, "F"] \arrow[d, hookrightarrow, swap, ""]  & X \arrow[d, "p"]  \\
\Delta^n \arrow[r, swap, ""] \arrow[ur, dashrightarrow, ""]  & Y
\end{tikzcd}\]
a dotted lift exists. 
\end{definition}

\begin{definition}
Let \(p : X \to Y\) be an inner fibration. Then \(p\) is a cocartesian fibration if for any edge \(\phi : y \to y'\) in \(Y_1\), and for every \(x \in X_0\) lying over \(y\), there exists a \(p\)-cocartesian edge \(f : x \to x'\) of \(X\) lying over \(\phi\). 
\end{definition}

This is a test to check pushing to GitHub is working as expected.

%\tableofcontents

%\pagebreak

\printbibliography

\end{document}